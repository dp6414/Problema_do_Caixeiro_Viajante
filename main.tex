\documentclass{article}
\usepackage{graphicx} 
\usepackage{titlesec}
\usepackage{amsmath}
\usepackage[absolute,overlay]{textpos}

\title{Problema do Caixeiro Viajante}
\author{David Gonçalves \\ Pedro Domingues Viana}

\date{Abril 2025}

\begin{document}
\begin{textblock*}{2cm}(15.1cm,0.5cm) % {largura}(x,y) em cm
\includegraphics[width=4cm]{ImagemEvento.ashx.jpeg}
\end{textblock*}
\maketitle
\newpage
\tableofcontents
\newpage

\section{Introdução}
O Problema do Caixeiro Viajante, conhecido em inglês por Travelling Salesman Problem (TSP), é um dos problemas mais clássicos e estudados no campo da otimização combinatória e das ciências da computação. A sua formulação é aparentemente simples: encontrar o caminho mais curto que permita a um vendedor visitar um conjunto de cidades exatamente uma vez e retornar à cidade de origem. No entanto, apesar da simplicidade da sua descrição, trata-se de um problema de elevada complexidade, cuja resolução ótima se torna impraticável à medida que o número de cidades aumenta.
O TSP não é apenas um exercício teórico, tem aplicações reais em áreas como logística, transportes, circuitos eletrónicos, biologia computacional e muito mais. A sua importância reside no facto de estar relacionado com diversos outros problemas de otimização e na sua utilidade como base para o desenvolvimento e teste de algoritmos.
Neste relatório, será feita uma abordagem detalhada ao problema, desde a sua definição e complexidade, até aos algoritmos mais comuns utilizados na tentativa de o resolver, tanto em termos exatos como aproximados. Como suporte visual e prático, foi desenvolvida uma aplicação interativa que permite simular diferentes estratégias de resolução sobre grafos personalizados.

\newpage

\section{Conceitos e Resultados de Teoria de Grafos}
\subsection{Definição de Grafo}
Um grafo é uma representação de um conjunto de pontos e do
modo como eles estão ligados. Aos pontos de um grafo chamamos vértices e ás
ligações entre eles, chamamos arestas.

\subsubsection{Grafo Ponderado}
Um grafo ponderado é um grafo onde cada aresta possui um valor específico sendo este o seu peso.

\subsubsection{Grafo Completo}
Um grafo completo é um grafo no qual dois quaisquer vértices são
adjacentes. Um grafo completo com n vértices representa-se por $K_n$.

\subsubsection{Grafo Hamiltoniano}
Chama-se caminho hamiltoniano a qualquer caminho elementar que passa por todos os vértices de $G$. Chama-se ciclo hamiltoniano a um ciclo que contém todos os vértices de $G$. Um grafo hamiltoniano é um grafo que contém um ciclo hamiltoniano.


\vspace{0.5cm}
\subsection{Definição de Caminho, Ciclo e Circuito}

\subsubsection{Caminho}
Um caminho num grafo $G$ é uma sequência de vértices de $G$ no
qual dois vértices sucessivos definem uma aresta.

\subsubsection{Ciclo}
Um ciclo é um circuito simples, não trivial, onde não há repetição de
vértices com a exceção dos vértices inicial e final.

\subsubsection{Circuito}
Um circuito é um caminho no qual o vértice inicial coincide com o vértice
final.

\vspace{0.5cm}

\subsection{Definição de Matriz de Adjacência e Matriz de Pesos}
\subsubsection{Matriz de Adjacência}
Uma matriz $[a_{ij}] \in \mathcal{M}_{n \times n}(\mathbb{Z})$ diz-se uma matriz de adjacência de $G$ se
    \[
a_{ij} =
\begin{cases}
1 & \text{se } v_i \text{ e } v_j \text{ são adjacentes}, \\
0 & \text{se } v_i \text{ e } v_j \text{ não são adjacentes}.
\end{cases}
\]

\subsubsection{Matriz de Pesos}
Uma matriz de pesos $[w_{ij}] \in \mathcal{M}_{n \times n}(\mathbb{R})$ representa os custos (ou distâncias, tempos, etc.) associados às arestas de um grafo ponderado. Assim, define-se:

\[
w_{ij} =
\begin{cases}
p & \text{se existe uma aresta entre } v_i \text{ e } v_j \text{ com peso } p, \\
0 & \text{se } v_i = v_j, \\
\infty & \text{se não existe aresta entre } v_i \text{ e } v_j.
\end{cases}
\]

\newpage

\section{Problema do Caixeiro Viajante}
\subsection{Descrição}
O Problema do Caixeiro Viajante consiste em encontrar o caminho mais curto que permita um "caixeiro" visitar um conjunto de cidades uma única vez, regressando ao ponto de partida.

\vspace{0.3cm}

Seja $G = (V, E)$ um grafo completo e ponderado, onde $V$ representa o conjunto de vértices (cidades) e $E$ o conjunto de arestas (ligações entre cidades), com pesos associados $w: E \rightarrow \mathbb{R}^+$ representando as distâncias.

\vspace{0.3cm}

O objetivo é minimizar a soma dos pesos das arestas percorridas, ou seja:
\[
\min \sum_{(v_i, v_j) \in C} w_{ij}
\]
onde $C$ é o ciclo hamiltoniano que visita todos os vértices exatamente uma vez.

\vspace{0.5cm}

\subsection{Variantes do TSP}
Preencher...

\vspace{0.5cm}

\subsection{Algorítmos}
\subsubsection{Vizinho mais Próximo}
\subsubsection{Mais Barato}
\subsubsection{Força Bruta}

\vspace{0.5cm}

\subsection{Dificuldades e Discussão sobre o TSP}

\vspace{0.5cm}

\subsection{Exemplos e Aplicações}

\vspace{0.5cm}

\subsection{Ponto de Vista da Computação}

\newpage

\section{Aplicação Desenvolvida}



\newpage

\section{Conclusão}

\newpage

\section{Bibliografia}


\end{document}
